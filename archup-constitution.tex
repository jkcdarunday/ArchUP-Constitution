\documentclass[12pt]{constitution}
% \usepackage{mathpazo}
\usepackage{setspace}
% \usepackage[utf8]{inputenc}
% \usepackage{indentfirst}

% \renewcommand{\baselinestretch}{2.0}

\begin{document}
\doublespacing
\title{The Constitution of Arch UP}
\author{The Charter Members of Arch UP}
\date{}
\maketitle
\setcounter{tocdepth}{0}
\tableofcontents
\newpage

We the members of Arch UP, a socio-civic organization of the University of the Philippines Los Ba\~{n}os,
open to novelties in the field of computing,
promoting friendly competition as one of our core values,
and instilling the sense of freedom,
do hereby ordain and promulgate this constitution.


\article{The Official Name of the Organization}

\section{Name.}
The official name of the organization shall be ``Arch UP''.

\section{Alternatives.}
Arch UP may also be referred to as ``Arch UPLB'', ``UPLB Arch'', ``UP Arch'', or simply ``Arch''.

\section{Acronyms.}
The acronym UP may be expanded to ``University of the Philippines'' and the acronym LB may be expanded to ``Los Ba\~{n}os''.

\article{The Official Seal of the Organization}

\article{Mission, Vision, and Principles of the Organization}

\section{Mission of the Organization.}
The organization aims to promote friendly competition as one of its core values and to instill freedom in the context
of freedom of information while pursuing camaraderie among fellow members.
 
\section{Vision of the Organization.}
The organization envisions a society whose sects improve upon each other through means of friendly competition while
instilling values for freedom, respect, and camaraderie.

\section{Principles of the Organization.}
The organization focuses primarily on the idea that friendly competition is paramount to
excellence may it be as a person, as a group, or as a nation.

The organization believes that every person has the ability to think, understand, and improve
through means of practice and training,
and that friendly competition is one way to achieve this goal.

\article{Membership}
\section{Limitations.}
Membership shall be open only to all bona fide students, employees and faculty members of the
University of the Philippines Los Ba\~{n}os.

\section{Discrimination.}
An applicant shall not be denied of acceptance into the organization 
may it be for the reason of sex, race, culture, religion, disabilities, and/or political belief and affiliation.

\section{Types of Membership}
The following are the different categories that a member may be categorized into:
\begin{enumerate}
  \item Resident -- members who are officially registered in any undergraduate, certificate, or post-graduate degree in
  the University of the Philippines Los Ba\~{n}os.

  \item Non-resident -- members who are registered students but are not currently enrolled in the University of the
  Philippines Los Ba\~{n}os.

  \item Inactive -- members who have officially filed inactivity that has been approved by the Executive Committee.

  \item Honorary -- members who do not qualify for the other provisions for membership but have been recognized
  for their considerable contribution to the organization and have been given approval by the majority of the
  current active resident and non-resident members.

  \item Alumni -- members who have finished their degree programs or have retired service to the University of
  the Philippines Los Ba\~{n}os.
\end{enumerate}

\article{Organizational Committees}
\section{Executive Committee.}
The executive committee is composed of
the president and the heads of other committees
and shall act as the leading committee of the organization
involving itself in all activities done by and within the organization.

\section{Secretarial Committee.}
The secretarial committee is in charge of taking down the minutes and attendance
during meetings, organizing documents and records, and processing of documents that are
required for recognition.

\section{Publicity Committee.}
The publicity committee is in charge of the dissemination of publicity materials, 
establishment of a good impression to the public,
management of relationships with other organizations,
and preparation of budget proposals, plans, and reports for its own activities.

\section{Membership Committee.}
The membership committee is in charge of
the management of the records of all members,
monitoring their participation to activities,
planning and execution of recruitment strategies,
formulation of the membership process,
and preparation of budget proposals, plans, and reports for its own activities.

\section{Activities Committee.}
The activities committee is in charge of
the planning, execution, and supervision of activities and projects,
assignation of roles for activities and projects to members,
and preparation of budget proposals, plans, and reports for its own activities.

\section{Commitee Heads.}
All committees shall have a committee head who will act as the primary 
representative of his or her own respective committee.

\article{Elections and Tenure}
\section{Nominations.}
Nominations shall be held in a general assembly or meeting at least a week before the elections.

\section{Inactivity.}
All nominees should have been active members of the organization for the past semester and
has been active in the organization for at least two semesters excluding summer.

\section{Length of Term.}
A nominee who will leave the university after at least 8 months of his term
through means including graduation may not be nominated for any position.

\section{Confirmation.}
Members who have been nominated but are absent during the meeting when
the nomination has been done will have to confirm his nomination
prior to the election, else, they will be disqualified

\section{Electoral System.}
The organization shall follow a majority-based electoral system.
In such case that no nominee has achieved more than fifty percent 
of the vote, all candidates that take up at least a part of the upper
fifty percent will be the new set of candidates until one nominee has
achieved more than fifty percent.

\section{Electoral Board.}
An electoral board shall be drafted from the list of active members
not exceeding one fifth of the number of active members. 
This board shall spearhead the planning and execution of the election.

\section{Approval.}
All plans that have been drafted by the electoral board shall require an
approval from the current executive committee.

\section{Vacancy.}
In the case that there is a vacancy in one of committee head positions,
an election will be done by the current members of the executive committee.

\article{Funds, Properties, Fees, and Dues}
\section{Sources of Funds}
The organization acquires its funds from sources including, but not limited to,
processing fees, semestral fees, fund raising activities, sponsorship,
revolving funds, and income from organizational projects.

\section{Processing Fees.}
Fees may be collected upon the filing a request for the change of membership status
including the registration of new recruits to the organization.

\section{Semestral Fees.}
Funds that are collected as fees during the beginning of every semester
are called Semestral Fees.

\section{Fund-Raising Activities.}
Activities by the Finance Committee that seeks to obtain funds through means
that are alternative tothe other presented sources of funds are called
Fund-Raising Activities.

\section{Sponsorship.}
Sponsorship are funds that originate from sponsors and may only be used as agreed upon with them.

\section{Revolving Funds.}
A certain amount that the majority of members have agreed upon will be collected during every official meeting.

\section{Project Income.}
A part of the income from internal and external projects may be used as funds for the organization.

\article{Impeachment and Disciplinary Action}
\section{Impeachment and Expulsion.}
Any member of any committee may be impeached or expulsed for reasons such as
the violation of the constitution, negligence, corruption, and grave misconduct.

\section{Inactivity.}
Inactivity shall be monitored, addressed, and evaluated by the
membership committee and approved by the executive committee.

\section{Resignation.}
Resignation shall be addressed and evaluated by the membership committee and
approved by the executive committee.
\end{document}
